\begin{tikzpicture}[every text node part/.style={align=center},initial text=$ $]
  \node (classical) {
    \begin{sequencediagram}
    \tikzstyle{inststyle}=[rectangle,anchor=west,minimum
    height=0.8cm, minimum width=1.6cm, fill=white]

    \newinst{f1}{Func 1}
    \newinst{f2}{Func 2}
    \newinst{f3}{Func 3}

    \begin{call}{f1}{call}{f2}{return}
    \begin{call}{f2}{call}{f3}{return}
    \end{call}
    \end{call}
  \end{sequencediagram}
};

\node[right=2.4cm of classical] (pipeline) {
  \begin{sequencediagram}
    \tikzstyle{inststyle}=[rectangle,anchor=west,minimum
    height=0.8cm, minimum width=1.6cm, fill=white]

    \newinst{f1}{Func 1}
    \newinst{f2}{Func 2}
    \newinst{f3}{Func 3}

    \mess{f1}{call}{f2};
    \mess{f2}{call}{f3};
    \mess{f3}{return}{f1};
  \end{sequencediagram}
};
\matrix[below=2cm of pipeline, matrix of nodes,, column sep=.5cm, row sep=.5cm] (chain) {
\node[initial, state] (cmain) {Func\\1}; &
\node[state] (cshort0) {Func\\2}; &
\node[state] (cshort1) {Func\\3}; \\
};
\matrix[left=2cm of chain, matrix of nodes,, column sep=0.5cm, row sep=.5cm] (star) {
\node[initial, state] (main) {Func\\1}; &
\node[state] (short0) {Func\\2};&
\node[state] (short1) {Func\\3}; \\
};
\draw[->] (cmain) -- (cshort0);
\draw[->] (cshort0) -- (cshort1);
\draw[->] (cshort1) to[out=225, in=315]  (cmain);

\draw[->] (main)   to[out=45, in=135] (short0);
\draw[->] (short0) to[out=45, in=135] (short1);
\draw[->] (short1) to[out=225, in=31
5] (short0);
\draw[->] (short0) to[out=225, in=315] (main);


\draw[->, very thick] (classical) --node[anchor=south] {tail\\recursion} (pipeline);
\draw[->, cery thick] (pipeline) --node[anchor=west] {communication\\graph} (chain);
\draw[->, cery thick] (classical) --node[anchor=west] {communication\\graph} ($(classical.south)-(0,1.7cm)$);
\end{tikzpicture}
